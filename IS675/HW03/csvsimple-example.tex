%% The LaTeX package csvsimple - version 1.12 (2014/07/14)
%% csvsimple-example.tex: an example for csvsimple
%%
%% -------------------------------------------------------------------------------------------
%% Copyright (c) 2008-2014 by Prof. Dr. Dr. Thomas F. Sturm <thomas dot sturm at unibw dot de>
%% -------------------------------------------------------------------------------------------
%%
%% This work may be distributed and/or modified under the
%% conditions of the LaTeX Project Public License, either version 1.3
%% of this license or (at your option) any later version.
%% The latest version of this license is in
%%   http://www.latex-project.org/lppl.txt
%% and version 1.3 or later is part of all distributions of LaTeX
%% version 2005/12/01 or later.
%%
%% This work has the LPPL maintenance status `author-maintained'.
%%
%% This work consists of all files listed in README
%%
\documentclass{article}
\usepackage{array,booktabs}
\usepackage{csvsimple}

\begin{document}

%----------------------------------------------------------
\section{Automatic table generation (for testing)}

{\small
\csvautotabular{csvsimple-example.csv}}


%----------------------------------------------------------
\section{My first CSV table}
\csvreader[tabular=|l|l|,
  table head=\hline\multicolumn{2}{|c|}{\bfseries My telephone book}\\\hline
             \bfseries Name & \bfseries Number\\\hline\hline,
  late after line=\\\hline]%
  {csvsimple-example.csv}{last name=\surname,first name=\givenname,telephone=\telephone}{%
    \givenname\ \surname & \telephone
}


%----------------------------------------------------------
\section{Remembering the names}
\csvnames{my names}{last name=\surname,first name=\givenname,address=\address,zip=\zip,telephone=\telephone,year of birth=\birthyear}

\csvreader[my names, late after line=\\, late after last line=]%
  {csvsimple-example.csv}{}{%
    \givenname\ was born in \birthyear\ and lives in \address.
}


%----------------------------------------------------------
\section{Filter fun}

\csvreader[my names, filter equal={\address}{Shrimpsbury}, tabbing,
  table head=\bfseries Shrimpsbury friends: \=\hspace*{3cm}\=\+\kill,
  before first line=\<\bfseries Shrimpsbury friends:\>]%
  {csvsimple-example.csv}{}{%
    \surname, \givenname \> \telephone
}


%----------------------------------------------------------
\section{More filter fun}

\csvreader[my names, filter=\birthyear<1980, centered tabular=rllr,
  table head=\multicolumn{4}{c}{\bfseries People born before 1980}\\\toprule
     \# & Name & Postal address & input line no.\\\midrule,
     late after line=\\, late after last line=\\\bottomrule]%
  {csvsimple-example.csv}{}{%
    \thecsvrow & \givenname\ \surname & \zip\ \address & \thecsvinputline
}


%----------------------------------------------------------
\section{Again and again}

\csvstyle{my table}{my names,
  centered tabular=|r|l|l|l|,
  table head=\hline\multicolumn{4}{|c|}{\bfseries #1}\\\hline
    \# & Name & Telephone & Postal address\\\hline\hline,
  late after line=\\, late after last line=\\\hline}

\csvreader[my table=Predefined table]{csvsimple-example.csv}{}{%
    \thecsvrow & \givenname\ \surname & \telephone & \zip\ \address
}

\csvreader[my table=Filtering for Smith, filter equal={\surname}{Smith}]%
    {csvsimple-example.csv}{}{%
    \thecsvrow & \givenname\ \surname & \telephone & \zip\ \address
}

\csvstyle{all and everything}{my table=#1, file={csvsimple-example.csv},
  command=\thecsvrow & \givenname\ \surname & \telephone & \zip\ \address}

\csvloop{all and everything=Loop instead of reader}

\csvloop{all and everything=With Shrimpsbury filter, filter equal={\address}{Shrimpsbury}}

\csvloop{all and everything=A little modification, late after line=\\\hline}


\end{document}

