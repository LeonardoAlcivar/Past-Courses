%%%%%%%%%%%%%%%%%%%%%%%%%%%%%%%%%%%%%%%%%
% KOMA-Script Presentation
% LaTeX Template
% Version 1.1 (18/10/15)
%
% This template has been downloaded from:
% http://www.LaTeXTemplates.com
%
% Original Authors:
% Marius Hofert (marius.hofert@math.ethz.ch)
% Markus Kohm (komascript@gmx.info)
% Described in the PracTeX Journal, 2010, No. 2
%
% License:
% CC BY-NC-SA 3.0 (http://creativecommons.org/licenses/by-nc-sa/3.0/)
%
%%%%%%%%%%%%%%%%%%%%%%%%%%%%%%%%%%%%%%%%%

%----------------------------------------------------------------------------------------
%	PACKAGES AND OTHER DOCUMENT CONFIGURATIONS
%----------------------------------------------------------------------------------------

\documentclass[
paper=128mm:96mm, % The same paper size as used in the beamer class
fontsize=11pt, % Font size
pagesize, % Write page size to dvi or pdf
parskip=half-, % Paragraphs separated by half a line
]{scrartcl} % KOMA script (article)

\linespread{1.12} % Increase line spacing for readability

%------------------------------------------------
% Colors
\usepackage{xcolor}	 % Required for custom colors
% Define a few colors for making text stand out within the presentation
\definecolor{mygreen}{RGB}{44,85,17}
\definecolor{myblue}{RGB}{34,31,217}
\definecolor{mybrown}{RGB}{194,164,113}
\definecolor{myred}{RGB}{255,66,56}
% Use these colors within the presentation by enclosing text in the commands below
\newcommand*{\mygreen}[1]{\textcolor{mygreen}{#1}}
\newcommand*{\myblue}[1]{\textcolor{myblue}{#1}}
\newcommand*{\mybrown}[1]{\textcolor{mybrown}{#1}}
\newcommand*{\myred}[1]{\textcolor{myred}{#1}}
%------------------------------------------------

%------------------------------------------------
% Margins
\usepackage[ % Page margins settings
includeheadfoot,
top=3.5mm,
bottom=3.5mm,
left=5.5mm,
right=5.5mm,
headsep=6.5mm,
footskip=8.5mm
]{geometry}
%------------------------------------------------

%------------------------------------------------
% Fonts
\usepackage[T1]{fontenc}	 % For correct hyphenation and T1 encoding
\usepackage{lmodern} % Default font: latin modern font
%\usepackage{fourier} % Alternative font: utopia
%\usepackage{charter} % Alternative font: low-resolution roman font
\renewcommand{\familydefault}{\sfdefault} % Sans serif - this may need to be commented to see the alternative fonts
%------------------------------------------------

%------------------------------------------------
% Various required packages
\usepackage{amsmath}
\usepackage{amssymb}
\usepackage{amsthm} % Required for theorem environments
\usepackage{bm} % Required for bold math symbols (used in the footer of the slides)
\usepackage{graphicx} % Required for including images in figures
\usepackage{tikz} % Required for colored boxes
\usepackage{booktabs} % Required for horizontal rules in tables
\usepackage{multicol} % Required for creating multiple columns in slides
\usepackage{lastpage} % For printing the total number of pages at the bottom of each slide
\usepackage[english]{babel} % Document language - required for customizing section titles
\usepackage{microtype} % Better typography
\usepackage{tocstyle} % Required for customizing the table of contents
%------------------------------------------------

%------------------------------------------------
% Slide layout configuration
\usepackage{scrpage2} % Required for customization of the header and footer
\pagestyle{scrheadings} % Activates the pagestyle from scrpage2 for custom headers and footers
\clearscrheadfoot % Remove the default header and footer
\setkomafont{pageheadfoot}{\normalfont\color{black}\sffamily} % Font settings for the header and footer

% Sets vertical centering of slide contents with increased space between paragraphs/lists
\makeatletter
\renewcommand*{\@textbottom}{\vskip \z@ \@plus 1fil}
\newcommand*{\@texttop}{\vskip \z@ \@plus .5fil}
\addtolength{\parskip}{\z@\@plus .25fil}
\makeatother

% Remove page numbers and the dots leading to them from the outline slide
\makeatletter
\newtocstyle[noonewithdot]{nodotnopagenumber}{\settocfeature{pagenumberbox}{\@gobble}}
\makeatother
\usetocstyle{nodotnopagenumber}

\AtBeginDocument{\renewcaptionname{english}{\contentsname}{\Large Outline}} % Change the name of the table of contents
%------------------------------------------------

%------------------------------------------------
% Header configuration - if you don't want a header remove this block
\ihead{
\hspace{-2mm}
\begin{tikzpicture}[remember picture,overlay]
\node [xshift=\paperwidth/2,yshift=-\headheight] (mybar) at (current page.north west)[rectangle,fill,inner sep=0pt,minimum width=\paperwidth,minimum height=2\headheight,top color=mygreen!64,bottom color=mygreen]{}; % Colored bar
\node[below of=mybar,yshift=3.3mm,rectangle,shade,inner sep=0pt,minimum width=128mm,minimum height =1.5mm,top color=black!50,bottom color=white]{}; % Shadow under the colored bar
shadow
\end{tikzpicture}
\color{white}\runninghead} % Header text defined by the \runninghead command below and colored white for contrast
%------------------------------------------------

%------------------------------------------------
% Footer configuration
\setlength{\footheight}{8mm} % Height of the footer
\addtokomafont{pagefoot}{\footnotesize} % Small font size for the footnote

\ifoot{% Left side
\hspace{-2mm}
\begin{tikzpicture}[remember picture,overlay]
\node [xshift=\paperwidth/2,yshift=\footheight] at (current page.south west)[rectangle,fill,inner sep=0pt,minimum width=\paperwidth,minimum height=3pt,top color=mygreen,bottom color=mygreen]{}; % Green bar
\end{tikzpicture}
\myauthor\ \raisebox{0.2mm}{$\bm{\vert}$}\ \myuni % Left side text
}

\ofoot[\pagemark/\pageref{LastPage}\hspace{-2mm}]{\pagemark/\pageref{LastPage}\hspace{-2mm}} % Right side
%------------------------------------------------

%------------------------------------------------
% Section spacing - deeper section titles are given less space due to lesser importance
\usepackage{titlesec} % Required for customizing section spacing
\titlespacing{\section}{0mm}{0mm}{0mm} % Lengths are: left, before, after
\titlespacing{\subsection}{0mm}{0mm}{-1mm} % Lengths are: left, before, after
\titlespacing{\subsubsection}{0mm}{0mm}{-2mm} % Lengths are: left, before, after
\setcounter{secnumdepth}{0} % How deep sections are numbered, set to no numbering by default - change to 1 for numbering sections, 2 for numbering sections and subsections, etc
%------------------------------------------------

%------------------------------------------------
% Theorem style
\newtheoremstyle{mythmstyle} % Defines a new theorem style used in this template
{0.5em} % Space above
{0.5em} % Space below
{} % Body font
{} % Indent amount
{\sffamily\bfseries} % Head font
{} % Punctuation after head
{\newline} % Space after head
{\thmname{#1}\ \thmnote{(#3)}} % Head spec
	
\theoremstyle{mythmstyle} % Change the default style of the theorem to the one defined above
\newtheorem{theorem}{Theorem}[section] % Label for theorems
\newtheorem{remark}[theorem]{Remark} % Label for remarks
\newtheorem{algorithm}[theorem]{Algorithm} % Label for algorithms
\makeatletter % Correct qed adjustment
%------------------------------------------------

%------------------------------------------------
% The code for the box which can be used to highlight an element of a slide (such as a theorem)
\newcommand*{\mybox}[2]{ % The box takes two arguments: width and content
\par\noindent
\begin{tikzpicture}[mynodestyle/.style={rectangle,draw=mygreen,thick,inner sep=2mm,text justified,top color=white,bottom color=white,above}]\node[mynodestyle,at={(0.5*#1+2mm+0.4pt,0)}]{ % Box formatting
\begin{minipage}[t]{#1}
#2
\end{minipage}
};
\end{tikzpicture}
\par\vspace{-1.3em}}
%------------------------------------------------

%----------------------------------------------------------------------------------------
%	PRESENTATION INFORMATION
%----------------------------------------------------------------------------------------

\newcommand*{\mytitle}{On the Uniqueness of the Shapley Value \\ by P. Dubey\cite{uniqueness-of-shapley-value}} % Title
\newcommand*{\runninghead}{} % Running head displayed on almost all slides
\newcommand*{\myauthor}{Terence Henriod} % Presenters name(s)
\newcommand*{\mydate}{\today} % Presentation date
\newcommand*{\myuni}{} % University or department

%----------------------------------------------------------------------------------------

\begin{document}

%----------------------------------------------------------------------------------------
%	TITLE SLIDE
%----------------------------------------------------------------------------------------

% Title slide - you may have to tweak a few of the numbers if you wish to make changes to the layout
\thispagestyle{empty} % No slide header and footer
\begin{tikzpicture}[remember picture,overlay] % Background box
\node [xshift=\paperwidth/2,yshift=\paperheight/2] at (current page.south west)[rectangle,fill,inner sep=0pt,minimum width=\paperwidth,minimum height=\paperheight/3,top color=mygreen,bottom color=mygreen]{}; % Change the height of the box, its colors and position on the page here
\end{tikzpicture}
% Text within the box
\begin{flushright}
\vspace{0.6cm}
\color{white}\sffamily
{\bfseries\Large\mytitle\par} % Title
\vspace{0.5cm}
\normalsize
\myauthor\par % Author name
\mydate\par % Date
\vfill
\end{flushright}

\clearpage

%----------------------------------------------------------------------------------------
%	TABLE OF CONTENTS
%----------------------------------------------------------------------------------------

\thispagestyle{empty} % No slide header and footer

\small\tableofcontents % Change the font size and print the table of contents - it may be useful to shrink the font size further if the presentation is full of sections
% To exclude sections/subsections from the table of contents, put an asterisk after \(sub)section like so: \section*{Section Name}

\clearpage

%----------------------------------------------------------------------------------------
%	PRESENTATION SLIDES
%----------------------------------------------------------------------------------------

%================================================
%------------------------------------------------

\section{Abstract}
\clearpage
%------------------------------------------------
%------------------------------------------------

\subsection*{A Unique Value for Superadditive Games}

Shapley showed that there is a unique value for the class of all superadditive games $D$ that follows certain, intuitive axioms

Shapley raised the question of whether an axiomatic foundation could be found for such a value in the context of the subclass $C$ or simple games alone.

This paper shows that this is possible.

\clearpage
%------------------------------------------------
%------------------------------------------------

\subsection*{Theorem 1: A proof of Shapley's Theorem for all games}

The paper give a new, simple proof of Shapley's Theorem for the set of all games $G$, not just superadditive ones

\clearpage
%------------------------------------------------
%------------------------------------------------

\subsection*{Theorem 2: A unique value using an axiomatic variant}

For the classes of games $C^{\prime}$ and $C^{\prime\prime}$, Shapley's axioms do not specify a unique value.

However, with some modification to one of the axioms (specifically the third), a unique value is obtained, and it happens to be the Shapley Value.

\clearpage
%------------------------------------------------
%------------------------------------------------

\subsection*{Notation}

N-player games may be referred to by their characteristic function, since that is the differentiating feature.

$\mathbb{R}$ and $\mathbb{Z}$ are the sets of Real and Integer numbers, respectively

Lowercase $s$, $t$, and $n$ might be used in place of $|S|, |T|,$ and $|N|$ respectively. Hopefully it is intuitive where this has occurred.

For a vector $v$, $v_{i}$ is the $i^{th}$ component of $v$.

$i$ is used as both a number and a name for the same player in $N$.

$\vee$ and $\wedge$ are the join and meet operations, respectively.

\clearpage
%------------------------------------------------

%================================================
%------------------------------------------------

\section{Introduction}
\clearpage
%------------------------------------------------
%------------------------------------------------

\subsection{A value for the power of players in $\mathbf{n}$-Player Cooperative Games}

An \textit{n-person cooperative game in characteristic function form} $G$ is defined as a pair
\[ G = (N, v)\]
where $N$ is a set of $n$ players
\[ N = (1, \dots, n) \]
and $v$ is a function
\[ v: 2^{N} \rightarrow \mathbb{R} \text{ such that } v(\emptyset) = 0 \]

\clearpage
%------------------------------------------------
%------------------------------------------------

\begin{itemize}
\item $v(S)$ represents the "worth"/"value"/"power" of coalition $S$
\item Given a game $G$, or more specifically a characteristic function $v$, it is desirable to know the "value" of each player in the game.
\end{itemize}

\clearpage
%------------------------------------------------
%------------------------------------------------


Let the set of all $n$ player games be $G$. We also refer to a game as its characteristic function $v$

Let $\phi$ be a function
\[ \phi: G \rightarrow \mathbb{R}^{n} \]
where $\phi_{i}(v)$ is the value of the $i^{th}$ player in the game.

\clearpage
%------------------------------------------------
%------------------------------------------------

\subsection{Preliminary Concepts}

\begin{enumerate}
\item A coalition $S$ is called a \emph{carrier} for the game if
	\[ v(T) = v(T \cap S) \qquad \forall \qquad T \subset N \]
    
\item If $\pi: N \rightarrow N$ is a permutation of $N$, then the game with characteristic function $\pi v$ is defined by
	\[ \pi_{v}(T) = v(\pi(T)) \qquad \forall \qquad T \subset N \]
    
\item Given any two games $v_{1}$, $v_{2}$,  $v_{1} + v_{2}$ is defined by
	\[ (v_{1} + v_{2})(T) = v_{1}(T) + v_{2}(T) \]
\end{enumerate}

\clearpage
%------------------------------------------------
%------------------------------------------------

\subsection{Three Axioms}

Shapley defined 3 axioms (in class we covered 4) that a $\phi$ should satisfy. 
\begin{enumerate}
\item If S is a carrier for $v$, then $\sum_{i \in S}{\phi_{i}(v)} = v(S)$
    
\item For any permutation $\pi$ and $i \in N$,
	\[ \phi_{\pi(i)}(\pi v) = \phi_{i}(v) \]
    
\item If $v_{1}$ and $v_{2}$ are any games, then
	\[ \phi(v_{1} + v_{2}) = \phi(v_{1}) + \phi(v_{2}) \]
\end{enumerate}

\clearpage
%------------------------------------------------
%------------------------------------------------

\subsection{Theorem 1}

\begin{theorem}[Shapley (1953)]
There is a unique function $\phi$, defined on $G$, which satisfies the axioms S1, S2, S3.
\end{theorem}
\clearpage
%------------------------------------------------
%------------------------------------------------

\subsection*{Proof of Theorem 1}

\begin{proof}
For each coalition $S$ and a constant $c$, define the game $v_{S, c}$ by
	\[ v_{S, c} = \begin{cases} 0 \text{ if } S \not\subset T \\
                                c \text{ if } S \subset T
                  \end{cases}\]
                  
Then $S$ and its supersets must all be carriers for $v_{S, c}$ (since only $S$ and its supersets can have an intersection with $S$ that is $S$, see the definition of a carrier)
\end{proof}
\clearpage
%------------------------------------------------
%------------------------------------------------

By the First Axiom, we have
	\[ \sum_{i \in S}{\phi_{i}(v_{S, c})} = c \]
and
	\[ \sum_{i \in S \cup \{j\}}{\phi_{i}(v_{S, c})} = c \text{ if } j \not\in S \]

Together, these imply
	\[ \phi_{j}(v_{S,c}) = 0 \text{ if } j \not\in S \]

\clearpage
%------------------------------------------------
%------------------------------------------------

By the Second Axiom, if $\pi$ is a permutation that interchanges $i$ and $j$ ($i, j \in S$) and leaves the other players fixed, then $\pi v_{S,c} = v_{S,c}$ (because the makeup of $S$ is unchanged), thus
	\[ \phi_{i}(v_{S,c}) = \phi_{j} (v_{S,c}) \text{ for any } i,j \in S \]

\clearpage
%------------------------------------------------
%------------------------------------------------

Therefore, $\phi_{v_{S,c}}$ is unique (since every way to "make" $v_{S,c}$ is the same one) if $\phi$ exists, and is given by
	\[ \phi_{i}(v_{S,c}) = \begin{cases} c / |S| &\text{ if } i \in S \\
                                         0       &\text{ if } i \not\in S
                           \end{cases} \]
because, the players of $S$ are all interchangeable, in a sense, they should have similar value

\clearpage
%------------------------------------------------
%------------------------------------------------

The games $\{v_{S, c} | S \neq \emptyset, S \subset N, c \in \mathbb{R}\}$ form an additive basis for the vector space $G$ (Shapley [1953] proves this).

For our purposes, we can consider the games
	\[ \{v^{\prime}_{S, c} | S \neq \emptyset, S \subset N, c \in \mathbb{R}\} \]
which are defined by
	\[ v^{\prime}_{S, c} = \begin{cases} 0 \text{ if } S \neq T \\
                                         c \text{ if } S = T
                  \end{cases}\]

Any game $v$ can be written as a finite sum of games of type $v^{\prime}_{S,c}$.

\clearpage
%------------------------------------------------
%------------------------------------------------

The uniqueness of $\phi$ follows, using the Third Axiom, if we can show that each $\phi(v^{\prime}_{S,c})$ is unique.

We can prove this by induction.

\clearpage
%------------------------------------------------
%------------------------------------------------

Assume that $\phi(v^{\prime}_{S,c})$ is unique for $|S| = k + 1, \dots, n$.

If we show that this is true for $|S| = k$ $\rightarrow$ mission accomplished.

Let $S_{1}, \dots S_{l}$ be all of the proper supersets of $S$.

Since $|S_{i}| > k, 1 \le i \le l$, $\phi(v^{\prime}_{S,c})$ is unique by the inductive assumption.

\clearpage
%------------------------------------------------
%------------------------------------------------

Since any game can be written as a sum of $v^{\prime}$ type games, We have
	\[ v_{S,c} = v^{\prime}_{S,c} + v^{\prime}_{S_{1},c} + \dots + v^{\prime}_{S_{l},c} \]
(Recall that the $v^{\prime}_{S,c}$ corresponds to our $k$ case.)

And so, by the Third Axiom, we have
	\[ \phi(v_{S,c}) = \phi(v^{\prime}_{S,c}) + \phi(v^{\prime}_{S_{1},c}) + \dots + \phi(v^{\prime}_{S_{l},c}) \]
    
Since we know that all of the terms in the sum except for $\phi(v^{\prime}_{S,c})$ are unique, then $\phi(v^{\prime}_{S,c})$ must also be unique.
    
\clearpage
%------------------------------------------------
%------------------------------------------------

Thus, $\phi$ satisfies all three axioms and is unique for any game in $G$ if it exists.

\clearpage
%------------------------------------------------
%------------------------------------------------

Implicit in the proof of uniqueness of $\phi$ is a recipe for constructing $\phi$.

Suppose that for $s = |S| = k + 1, \dots, n$
	\[ \phi_{i}(v_{S,c}) = \begin{cases}
                               \frac{(s - 1)!(n - s)!}{n!} * c &\text{ if } i \in S \\
    				           - \frac{s}{n - s} \frac{(s - 1)!(n - s)!}{n!} * c &\text{ if } i \not\in S
                  \end{cases} \]
                  
Using
\[ \phi(v_{S,c}) = \phi(v^{\prime}_{S,c}) + \phi(v^{\prime}_{S_{1},c}) + \dots + \phi(v^{\prime}_{S_{l},c}) \]

it follows that for $|S| = k$
\[ \phi_{i}(v^{\prime}_{S,c}) = \begin{cases}
                               \frac{(s - 1)!(n - s)!}{n!} * c &\text{ if } i \in S \\
    				           - \frac{s}{n - s} \frac{(s - 1)!(n - s)!}{n!} * c &\text{ if } i \not\in S
                  \end{cases} \]

\clearpage
%------------------------------------------------
%------------------------------------------------

Now, it is straightforward to obtain $\phi(v)$ for any $v$.

Since
	\[ v = \sum_{S \subset N: S \neq \emptyset}{v^{\prime}_{S, v(S)}} \]

then by the Third Axiom
	\[ \phi{v} = \sum_{S \subset N: S \neq \emptyset}{\phi(v^{\prime}_{S, v(S)})} \]

\clearpage
%------------------------------------------------
%------------------------------------------------

Simplifying the right hand side we have
	\[ \phi_{i}(v) = \sum_{i \in T, T\subset N}{\frac{(t - 1)!(n - t)!}{n!} [ v(T) - v(T - \{i\})} ] \]

This is the Shapley value (perhaps written differently than we saw in class, but it is the same).

We can verify that this formula satisfies the three axioms (which we did do in class).

\clearpage
%------------------------------------------------
%------------------------------------------------

\section{Uniqueness for Subclasses}

(Major takeaways:
\begin{itemize}
\item Since the Shapley Value was show to exist on G, we just need to show its uniqueness.

\item Games can be represented by linear combinations of games of type $v_{S,c}$
\end{itemize}

)

\clearpage
%------------------------------------------------
%------------------------------------------------

\subsection{Subclasses of $G$}

What if we want to apply the Three Axioms to subclasses of $G$, call them $K$.

Then the Second Axiom is only required to hold for $\pi v \in K \text{ if } v \in K$.

The Third Axiom is only required to hold if $v_{1} + v_{2} \in K \text{ if } v_{1}, v_{2} \in K$

But can the Axioms, restricted in this way, specify a unique $\phi$ for $K$?

\clearpage
%------------------------------------------------
%------------------------------------------------

Clearly, by looking at the Shapley value on $G$, we can see that there is at least one $\phi$ for these subclasses, but is it the only one?

For any $K$, we can use a procedure similar to the previous proof, we can establish the uniqueness of $\phi$ on $K$.

It is not necessary to do so by starting with $\phi$ defined on $G$ and restricting its domain to $K$.

\clearpage
%------------------------------------------------
%------------------------------------------------

\subsection{Case: $K = D$}

Consider the case $K = D$, where $D$ is the subclass of $G$ representing all superadditive games in $G$.

Recall that a superadditive game is one for which 
	\[ v(S \cup T) \ge v(S) + v(T) \text{ for } S \cap T = \emptyset \]

\clearpage
%------------------------------------------------
%------------------------------------------------

Start by demonstrating that a basis for $G$ is formed by
	\[ \{v_{S,c} | S \neq \emptyset, S \subset N, c \in \mathbb{R} \} \]

Suppose that $v^{\prime}_{S,1}$ is in the linear span of
	\[ \{v_{T, 1}| T \supset S \} \text{ when } |S| = k + 1, \dots, n \]    

\clearpage
%------------------------------------------------
%------------------------------------------------

Let $|S^{*} = k|$.

Since
	\[ v^{\prime}_{S^{*},1} = v_{S^{*},1} - v^{\prime}_{S^{*}_{1},1} - \dots - v^{\prime}_{S^{*}_{},1} \]
    
where $S^{*}_{1}, \dots, S^{*}_{j}$ are all proper supersets of $S^{*}$ and since by our inductive assumption each $v_{S^{*}_{i},1}$ is in the linear span of $\{v_{T, 1}| T \supset S^{*}_{i} \}$, it follows that $v^{\prime}_{S^{*},1}$ is in the linear span of $\{v_{T, 1}| T \supset S^{*}_{i} \}$.

\clearpage
%------------------------------------------------
%------------------------------------------------

Knowing that $\{v^{\prime}_{S,1} | S \neq \emptyset, S \subset N \}$ spans $G$, we can see that $\{v_{S,1} | S \neq \emptyset, S \subset N \}$ also spans $G$.

In fact, $\{v_{S,1} | S \neq \emptyset, S \subset N \}$ also spans $G$ because
\[ |\{v_{S,1} | S \neq \emptyset, S \subset N \}| =
	|\{v^{\prime}_{S,1} | S \neq \emptyset, S \subset N \}| \]
which already was known to be a basis of $G$.


\clearpage
%------------------------------------------------
%------------------------------------------------

We can express a game in $D$ uniquely as $v = \sum{c_{S}v_{S,1}}$.

Since some of the $c_{S}$ terms could be negative, this would mean that the game is not in $D$, and we can't apply the Third Axiom.

But if we transpose the terms with a negative $c_{S}$ coefficient to the left, we can see that the new equation contains only games in $D$. We can now apply the Third Axiom and prove the uniqueness of $\phi$ on $D$.

\clearpage
%------------------------------------------------
%------------------------------------------------

To find $c_{S}$ explicitly, express each $v^{\prime}_{S,1}$ in terms of the basis
	\[ \{ v_{S,1} | S \neq \emptyset, S \subset N \} \]
using
	\[ v^{\prime}_{S^{*},1} = v_{S^{*},1} - v^{\prime}_{S^{*}_{1},1} - \dots - v^{\prime}_{S^{*}_{},1} \]
and induction, then substitute into
	\[ v = \sum_{T: T\neq \emptyset, T \subset N}{v^{\prime}_{T, v(T)}} \]
 
This way, it can be shown that $c_{S} = \sum_{T \subset S}{(-1)^{s - t} v(T)}$

\clearpage
%------------------------------------------------
%------------------------------------------------

This result allows us to write out an explicit formula for $\phi(v)$, as did Shapley with his Value.

This is not simple though, and it is easier to show by restricting a $\phi$ found on $G$ to $D$

\clearpage
%------------------------------------------------
%------------------------------------------------

\subsection{Case: $K \neq D$}

The following examples (which are not exhaustive), we can prove the uniqueness of $\phi$ on that each class $K$ the same way we did for $G$ in Theorem 1, and construct $phi$ recursively in a similar manner.

\begin{enumerate}
\item $K = \{v | v(S) = 0 \text{ or } 1, \quad \forall \quad S \subset N\}$, i.e. \emph{simple games}.

\item $K = \{v | v(S) = 0 \text{ if } |S| \le k \text{ for some } k\}$, as well as all simple games with this restriction.

\item The subclass of games where certain players $i_{1}, \dots, i_{k}$ are distinguished and $v(S) = 0 \text{ if } \{i_{1}, \dots, i_{k}\} \not\subset S$, as well as all simple games with this restriction.
\end{enumerate}

\clearpage
%------------------------------------------------
%------------------------------------------------

\subsection{\emph{Remarks}}

\begin{enumerate}
\item The convex cone generated by simple games with veto players (players $i$ such that $v(S) = 0 \text{ if } i \not\in S \quad \forall \quad S \subset N$) is the subclass $L$ of all games with non-empty cores. Therefore, case $C$ shows that the axioms specify the Shapley value on $L$. In fact, this is true for convex cones generated by the class of games in any one of $A, B, C$ or any of their unions.

\item For any $P \subset G, |P| < \inf$, we can determine a finite number of steps whether or not the axioms uniquely specify the Shapley value on $P$; and if they do not, we can construct different $\phi$s on $P$ which do satisfy the axioms. This corresponds to checking if a system of linear equations has a unique solution. The size of the linear system can be reduced with a procedure mimicing the proof of Theorem 1 (not discussed).
\end{enumerate}

\clearpage
%------------------------------------------------

%================================================
%------------------------------------------------

\section{Monotonic Simple Games}

\clearpage
%------------------------------------------------
%------------------------------------------------

\subsection{Definitions}

Recall that $C$ is the set of all simple games.

Let $C^{\prime}$ be the subclass of all \emph{monotonic simple games} in $G$, i.e., games for which $v(S) = 1 \implies v(T) = 1, \quad \forall \; S \subset T$.

Let $C^{\prime\prime}$ be the subclass of all \emph{superadditive simple games} in $G$.

Note that $C^{\prime\prime} \subset C^{\prime}$.

\clearpage
%------------------------------------------------
%------------------------------------------------

\subsection{An Example of the Insufficiency of the Three Axioms}

The Three Axioms \emph{do not} uniquely specify the Shapley value on $C^{\prime}$ or $C^{\prime\prime}$ if $|N| > 2$.

There are games in $C^{\prime}$ and $C^{\prime\prime}$ for which the Shapley Value is determined by the First and Second Axioms, the games of type $v_{S,1}$.

However, we want to define the Shapley Value for all of these games

\clearpage
%------------------------------------------------
%------------------------------------------------

Take a game that is not of type $v_{S,1}$ in $C^{\prime\prime}$ (and thus $C^{\prime}$ also). An example is:
	\[ v(N - \{i\}) = v(N - \{j\}) = v(N) = 1 \]
    \[ v(S) = 0 \text{ for all other } S \]
    \[ i, j \in N, i \neq j, |N| > 2 \]

Let $\phi$ be the following:
	\[ \phi_{i}(v) = \phi_{j}(v) = p \text{ for } p \in \mathbb{R} \]
    \[ \phi_{k}(v) = \frac{1 - 2p}{|N - \{i, j\}|} \text{ for } k \neq i, j \]

\clearpage
%------------------------------------------------
%------------------------------------------------

The game satisfies the First Axiom because 
	\[ \sum_{i \in N}{\phi_{i}(v)} = v(N) \]

The game satisfies the Second Axiom because players maintain the same value, even if the game is permuted.

\clearpage
%------------------------------------------------
%------------------------------------------------

This $\phi$ also satisfies the Third Axiom vacuously. Suppose
	\[ v + v^{\prime} = v^{\prime\prime} \text{ for } v^{\prime}, v^{\prime\prime} \in C^{\prime} \]
    
But
	\[v(N) = 1 \implies v^{\prime\prime} = 1 \implies v^{\prime}(N) = 0 \implies v^{\prime} = 0 \text{ (because it is monotonic)}\]

Also, if 
	\[ v - v^{\prime} = v^{\prime\prime} \text{ for } v^{\prime}, v^{\prime\prime} \in C^{\prime} \]

then two cases arise...

\clearpage
%------------------------------------------------
%------------------------------------------------

\begin{enumerate}
\item $v^{\prime\prime}(N) = 1$ \\
$v^{\prime}(N) \text{ must } = 0 \implies v^{\prime} = 0$

\item $v^{\prime\prime}(N) = 1$ \\
Therefore $v^{\prime\prime} = 0$ and $v^{\prime\prime} = v^{\prime}$
\end{enumerate}

The Third Axiom is satisfied for any value of $p$, so $\phi$ is not uniquely specified on $C^{\prime}$ or $C^{\prime\prime}$ by the Three Axioms.

\clearpage
%------------------------------------------------
%------------------------------------------------

\subsection{A Variant of the Third Axiom}

We can replace the Third Axiom with a variant, and  $\phi$ will be uniquely specified on $C^{\prime}$ or $C^{\prime\prime}$, and it will be the Shapley Value (given later).

Note that as we construct the variant, we will do so for $C^{\prime\prime}$. The construction is the same for $C^{\prime}$.

\clearpage
%------------------------------------------------
%------------------------------------------------

Let us start with some definitions. Consider the games $v, v^{\prime} \in C^{\prime\prime}$:

Define the join of the two games ($v \vee v^{\prime}$) to be
\[(v \vee v^{\prime})(S) = \begin{cases}
                             1 \text{ if either } v(S) = 1 \text{ or } v^{\prime}(S) = 1 \\
                             0 \text{ if } v(S) = 0 \text{ and } v^{\prime}(S) = 0
                             \end{cases} \]

Define the join of the two games ($v \wedge v^{\prime}$) to be
\[(v \wedge v^{\prime})(S) = \begin{cases}
                             1 \text{ if } v(S) = 1 \text{ and } v^{\prime}(S) = 1 \\
                             0 \text{ if either } v(S) = 0 \text{ or } v^{\prime}(S) = 0
                             \end{cases} \]
                             
Note that $v \vee v^{\prime}$ may not always be in $C^{\prime\prime}$, but that $v \wedge v^{\prime}$ is, even for $v, v^{\prime} \in C^{\prime}$.
                             
\clearpage
%------------------------------------------------
%------------------------------------------------

We can verify that $v \wedge v^{\prime} \in C^{\prime\prime} \text{ for } v, v^{\prime} \in C^{\prime\prime}$.

(By Contradiction) Suppose that $v \wedge v^{\prime} \not\in C^{\prime\prime}$.

Then there are coalitions $S, T \text{ s.t. } S \cap T = \emptyset$ and
	\[ (v \wedge v^{\prime})(S \cup T) < (v \wedge v^{\prime})(S) + (v \wedge v^{\prime})(T) \]
    
But by the definition of $v \wedge v^{\prime}$ this means that either
	\[ v(S \cup T) < v(S) + v(T) \text{ or} \]
	\[ v^{\prime}(S \cup T) < v^{\prime}(S) + v^{\prime}(T) \text{ or} \]
    
Which is a contradiction. (Consider the possible values for the various terms.)
    
\clearpage
%------------------------------------------------
%------------------------------------------------

The Variant of the Third Axiom can be stated as:

If $v \vee v^{\prime} \in C^{\prime\prime}$ whenever $v, v^{\prime} \in C^{\prime\prime}$ then
\[ \phi(v \vee v^{\prime}) + \phi(v \wedge v^{\prime}) = \phi(v) + \phi(v^{\prime}) \]

(For the $C^{\prime}$ version, we may drop the if because $v \vee v^{\prime} \in C^{\prime}$ always.)

\clearpage
%------------------------------------------------
%------------------------------------------------

\subsection{Theorem 2: A Variant of the Third Axiom (Finally)}

\begin{theorem}[Dubey 1975]
There is a unique function $\phi$, defined on $C^{\prime\prime}$, which satisfies the Three Axioms (the third one being our variant). Moreover, $\phi$ is the Shapley Value.
\end{theorem}

\clearpage
%------------------------------------------------
%------------------------------------------------

\subsection*{Proof of Theorem 2}

Every $v \in C^{\prime\prime}$ has a finite number of minimal winning coalitions $S_{1}, \dots S_{k}$, i.e., coalitions $S_{i}$ such that $v(T) = 1$ if $S_{i} \subset T$ for some $i$ and $v(T) = 0$ if $S_{i} \not\subset T \quad \forall \; i$.

Clearly $v = v_{S_{1},1} \vee \dots \vee v_{S_{k},1}$, where the right side is defined associatively.

Let 
\[ n^{1}(v) = min \{p \in \mathbb{Z}^{+} | \exists \; \text{a minimal winning coalition } T \text{ with } |T| = p \} \text{ and}\]
\[n^{2}(v) = \text{the number of winning coalitions } T \text{ such that } |T| = n^{1}(v)\]

\clearpage
%------------------------------------------------
%------------------------------------------------

The proof of uniqueness of $\phi$ will be on induction of $n^{1}(v)$ and $n^{2}(v)$.

For the case where $n^{1}(v) = n, v = v_{N, 1}$, $\phi$ is obviously unique.

Assume that $\phi(v)$ is unique for all $v$ such that $n^{1}(v) = k + 1, \dots, n$. Then $\phi$ is unique when $n^{1}(v) = k$ and $n^{2}(v) = 1$

Let $S$ be the unique minimal winning coalition with $k$ players. If $S$ is the only minimal winning coalition of $v$, then $v = v_{S,1}$ and $\phi(v)$ is unique. Otherwise, let $S_{1}, \dots S_{m}$ denote all of the minimal winning coalitions that aren't $S$.

Note: $|S_{i}| > k$ for $1 \le i \le m$ since $n^{2}(v) = 1$.

\clearpage
%------------------------------------------------
%------------------------------------------------

Now,
\[ v = (v_{S_{1},1} \vee \dots \vee v_{S_{m},1}) \vee v_{S,1} \]
We restate this as
\[ v = v^{\prime} \vee v_{S,1} \]

It follows that $n^{1}(v^{\prime}) > k$. Therefore $\phi(v^{\prime})$ is unique by the inductive assumption

Further, $n^{1}(v_{S,1} \wedge v^{\prime}) > k$. This is apparent from the definition of $\wedge$. Therefore $\phi(v \vee v^{\prime})$ is also unique by the inductive assumption. Invoke our Third Axiom Variant, then
\[ \phi(v) = \phi(v^{\prime} \vee v_{S,1}) = \phi(v^{\prime}) + \phi(v_{S,1}) - \phi(v_{S,1} \wedge v^{\prime}) \]

Since all three vectors on the right side are unique, so is $\phi(v)$.

\clearpage
%------------------------------------------------
%------------------------------------------------

Next, suppose that $\phi(v)$ has been shown to be unique for all $v$ such that either
\[ n^{1}(v) = k + 1, \dots n \text{ or}\]
\[ n^{1}(v) = k \text{ and } n^{2}(v) = 1, \dots, j \]

Then $\phi(v)$ is unique if we can show that it is so for $n^{1}(v) = k$ and $n^{2}(v) = j + 1$.

\clearpage
%------------------------------------------------
%------------------------------------------------

Let $S_1, \dots, S_{j + 1}$ be the minimal winning coalitions of $v$ with $k$ players each.

Let $T_{1}, \dots, T_{m}$ be all the other minimal winning coalitions of $v$. By the conditions on $n^{1}(v)$ and $n^{2}(v)$ it is clear that $|T_{i}| > k$ for $1 \le i \le m$.

Now,
\[ v = (v_{T_{1},1} \vee \dots \vee v_{T_{m},1} \vee v_{S_{1},1} \vee \dots \vee v_{S_{j},1}) \vee v_{S_{j+1,1}} \]
We abbreviate this as
\[ v = v^{\prime\prime} \vee v_{S_{j+1},1} \]

\clearpage
%------------------------------------------------
%------------------------------------------------

We can now see that $v^{\prime\prime}$ satisfies $n^{1}(v) = k \text{ and } n^{2}(v) = 1, \dots, j$

and $v^{\prime\prime} \wedge v_{S_{j+1,1}}$ satisfies $n^{1}(v) = k + 1, \dots n$. 

Therefore, both $\phi(v^{\prime\prime})$ and $\phi(v^{\prime\prime} \wedge v_{S_{j+1},1})$ are unique by the inductive assumption.

By our Third Axiom Variant, we have
\[ \phi(v) = \phi(v^{\prime\prime} \vee v_{S_{j+1},1}) = \phi(v_{S_{j+1},1}) - \phi(v^{\prime\prime} \wedge v_{S_{j+1},1}) \]

which proves the uniqueness of $\phi$.

\clearpage
%------------------------------------------------
%------------------------------------------------

Combining the two results, we see that $\phi(v)$ is unique fr any feasible numbers $n^{1}(v)$ and $n^{2}(v)$, i.e., for all $v \in C^{\prime\prime}$.

It is clear that the Shapley Value $\phi$ on $G$ satisfies the Three Axioms (with our variant) when it is restricted to $C^{\prime\prime}$.

Indeed, $v + v^{\prime} = (v \vee v^{\prime}) + (v \wedge v^{\prime})$ when we consider the $+$ operation to occur in the space of $G$. Hence, by the Third Axiom, $\phi(v) + \phi(v^{\prime}) = \phi(v \vee v^{\prime}) + \phi(v \wedge v^{\prime})$.

Thus, the Shapley Value is \emph{the} unique $\phi$ on $C^{\prime\prime}$ which satisfies the Three Axioms (with our variant).

\clearpage
%------------------------------------------------
%------------------------------------------------

Note that we don't need to depend on the $\phi$ defined on $G$ to establish the existence of $\phi$ on $C^{\prime\prime}$. The implicit proof of uniqueness of $\phi$ on $C^{\prime\prime}$ is a straightforward recursive construction.

\clearpage
%------------------------------------------------
%------------------------------------------------

\subsection{Remarks}

\begin{enumerate}
\item Theorem 2 holds when we replace $c^{\prime\prime}$ with $C^{\prime}$. The proofs are similar and involve stopping the induction at appropriate stages, considering games that take on values in $\{c, 0\}$ instead of $\{1, 0\}$.

Two examples are sub-classes of $C^{\prime}$ (or $C^{\prime\prime}$) for which:
	\begin{enumerate}
    \item $v(S) = 0$ if $|S| \le k$ (and $1$ otherwise)
    \item $v(S) = 0$ if $\{i_{1}, \dots, i_{k}\} \not\subset S$ (and $1$ otherwise)
	\end{enumerate}

\item By changing the First Axiom, but retaining the Second and our variant of the Third, we can obtain an axiomatic value for the \emph{Banzhaff Value} in its unnormalized form [Lucas, 1973] when it is restricted to $C^{\prime}$ or $C^{\prime\prime}$. This proof is similar to the proof of Theorem 2, and appears in a later paper.

\end{enumerate}

\clearpage
%------------------------------------------------
%------------------------------------------------

\thispagestyle{empty} % No slide header and footer

\bibliographystyle{unsrt}
\bibliography{presentation_2}

\clearpage

%------------------------------------------------

\thispagestyle{empty} % No slide header and footer

\begin{tikzpicture}[remember picture,overlay] % Background box
\node [xshift=\paperwidth/2,yshift=\paperheight/2] at (current page.south west)[rectangle,fill,inner sep=0pt,minimum width=\paperwidth,minimum height=\paperheight/3,top color=mygreen,bottom color=mygreen]{}; % Change the height of the box, its colors and position on the page here
\end{tikzpicture}
% Text within the box
\begin{flushright}
\vspace{0.6cm}
\color{white}\sffamily
{\bfseries\LARGE Thank you!\par} % Request for questions text
\vfill
\end{flushright}

%----------------------------------------------------------------------------------------

\end{document}