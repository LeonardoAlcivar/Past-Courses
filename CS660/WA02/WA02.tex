\documentclass[fleqn]{article}
\usepackage[utf8]{inputenc}
\usepackage{graphicx}
  \DeclareGraphicsExtensions{.png, .jpeg}
\usepackage[top=1in, bottom=1in, left=1in, right=1in]{geometry}

\usepackage{parsetree}
\usepackage{amsmath}

\title{Compiler Construction \\ WA02: Parse Trees}
\author{Terence Henriod}
\date{\today}

\begin{document}

\clearpage            % All of
\maketitle            % this,
\thispagestyle{empty} % removes the page number from the title page

\begin{abstract}
\noindent This assignment asks you to prepare written answers to questions on context-free grammars.
Each question has a short answer. You may discuss this assignment with other
students and work the problems together. However, your writeup should be your own individual
work. Remember written assignments are to be turned in in class on the date
due.
\end{abstract}

\newpage

\begin{enumerate}
  \item Given the Grammar G =
  \begin{itemize}
    \item $A \rightarrow Ba\; |\; bC$
    \item $B \rightarrow d\; |\; eBf$
    \item $C \rightarrow gC\; |\; g$
  \end{itemize}
  Determine which strings are in L(G). Construct parse trees for those that are.

  \begin{itemize}
    \item bg\\\\
    \textit{Answer}:\\
    \begin{parsetree}
      (.A.
        `b'
        (.C.
          `g'
        )
      )
    \end{parsetree}\\

    \item bffd\\\\
    \textit{Answer}:\\
    Not in L(G).
    \begin{parsetree}
      (.A.
        `b'
        (.C.
          `No way to get f or d'
        )
      )
    \end{parsetree}\\

    \item bggg\\\\
    \textit{Answer}:\\
    \begin{parsetree}
      (.A.
        `b'
        (.C.
          `g'
          (.C.
            `g'
            (.C.
              `g'
            )
          )
        )
      )
    \end{parsetree}\\

    \item edfa\\\\
    \textit{Answer}:\\
    \begin{parsetree}
      (.A.
        (.B.
          `e'
          (.B.
            `d'
          )
          `f'
        )
        `a'
      )
    \end{parsetree}\\

    \hfill\newline\hfill\newline
    \item eedffa\\\\
    \textit{Answer}:\\
    \begin{parsetree}
      (.A.
        (.B.
          `e'
          (.B.
            `e'
            (.B.
              `d'
            )
            `f'
          )
          `f'
        )
        `a'
      )
    \end{parsetree}\\

    \item faae\\\\
    \textit{Answer}:\\
    Not in L(G).
    \begin{parsetree}
      (.A.
        `Given string does not end with a or start with b'
      )
    \end{parsetree}\\

    \item defa\\\\
    \textit{Answer}:\\
    Not in L(G).
    \begin{parsetree}
      (.A.
        (.B.
          `Rewrite to d $\rightarrow$ no ef'
          `Rewrite to eBf $\rightarrow$ no d before ef'
        )
        `a'
      )
    \end{parsetree}\\
  \end{itemize}

  \newpage
  \item Given the following parse tree,
  \begin{parsetree}
    ( .S.
      (.A.
        (.B.
          `e'
          `f'
        )
        `d'
      )
      (.B.
        `e'
        `f'
      )
      (.C.
        `g'
        (.C.
          (.A.
            `h'
          )
          `d'
        )
        `g'
      )
    )
  \end{parsetree}\\

  \begin{enumerate}
    \item Construct the corresponding rightmost derivation.\\\\
    \textit{Answer}:\\
    (Both left and right derivations will be shown simultaneously, along with the sentential form of the string they produce)\\

    \underline{Left} \hspace{0.4\textwidth} \underline{Right}\\\\
    
    S  \hspace{0.42\textwidth} S\\
    \begin{parsetree}
      (`S')
    \end{parsetree}
    \hspace{0.4\textwidth}
    \begin{parsetree}
      (`S')
    \end{parsetree}\\\\

    ABC \hspace{0.4\textwidth} ABC\\
    \begin{parsetree}
      (.S.
        `A'
        `B'
        `C'
      )
    \end{parsetree}
    \hspace{0.35\textwidth}
    \begin{parsetree}
      (.S.
        `A'
        `B'
        `C'
      )
    \end{parsetree}\\\\
    
    BdBC \hspace{0.4\textwidth} ABgCg\\
    \begin{parsetree}
      (.S.
        (.A.
          `B'
          `d'
        )
        `B'
        `C'
      )
    \end{parsetree}
    \hspace{0.30\textwidth}
    \begin{parsetree}
      (.S.
        `A'
        `B'
        (.C.
          `g'
          `C'
          `g'
        )
      )
    \end{parsetree}\\\\
    
    efdBC \hspace{0.4\textwidth} ABgAdg\\
    \begin{parsetree}
      (.S.
        (.A.
          (.B.
            `e'
            `f'
          )
          `d'
        )
        `B'
        `C'
      )
    \end{parsetree}
    \hspace{0.28\textwidth}
    \begin{parsetree}
      (.S.
        `A'
        `B'
        (.C.
          `g'
          (.C.
            `A'
            `d'
          )
          `g'
        )
      )
    \end{parsetree}\\\\
    
    \newpage
    efdefC \hspace{0.4\textwidth} ABghdg\\
    \begin{parsetree}
      (.S.
        (.A.
          (.B.
            `e'
            `f'
          )
          `d'
        )
        (.B.
          `e'
          `f'
        )
        `C'
      )
    \end{parsetree}
    \hspace{0.26\textwidth}
    \begin{parsetree}
      (.S.
        `A'
        `B'
        (.C.
          `g'
          (.C.
            (.A.
              `h'
            )
            `d'
          )
          `g'
        )
      )
    \end{parsetree}\\\\
  
    efdefgCg \hspace{0.4\textwidth} Aefghdg\\
    \begin{parsetree}
      (.S.
        (.A.
          (.B.
            `e'
            `f'
          )
          `d'
        )
        (.B.
          `e'
          `f'
        )
        (.C.
          `g'
          `C'
          `g'
        )
      )
    \end{parsetree}
    \hspace{0.2\textwidth}
    \begin{parsetree}
      (.S.
        `A'
        (.B.
          `e'
          `f'
        )
        (.C.
          `g'
          (.C.
            (.A.
              `h'
            )
            `d'
          )
          `g'
        )
      )
    \end{parsetree}\\\\
  
    efdefgAdg \hspace{0.4\textwidth} Bdefghdg\\
    \begin{parsetree}
      (.S.
        (.A.
          (.B.
            `e'
            `f'
          )
          `d'
        )
        (.B.
          `e'
          `f'
        )
        (.C.
          `g'
          (.C.
            `A'
            `d'
          )
          `g'
        )
      )
    \end{parsetree}
    \hspace{0.12\textwidth}
    \begin{parsetree}
      (.S.
        (.A.
          `B'
          `d'
        )
        (.B.
          `e'
          `f'
        )
        (.C.
          `g'
          (.C.
            (.A.
              `h'
            )
            `d'
          )
          `g'
        )
      )
    \end{parsetree}\\\\
    
    efdefghdg \hspace{0.4\textwidth} efdefghdg\\
    \begin{parsetree}
      (.S.
        (.A.
          (.B.
            `e'
            `f'
          )
          `d'
        )
        (.B.
          `e'
          `f'
        )
        (.C.
          `g'
          (.C.
            (.A.
              `h'
            )
            `d'
          )
          `g'
        )
      )
    \end{parsetree}
    \hspace{0.1\textwidth}
    \begin{parsetree}
      (.S.
        (.A.
          (.B.
            `e'
            `f'
          )
          `d'
        )
        (.B.
          `e'
          `f'
        )
        (.C.
          `g'
          (.C.
            (.A.
              `h'
            )
            `d'
          )
          `g'
        )
      )
    \end{parsetree}\\\\

    \newpage
    \item Construct as much of the grammar as can be determined from the parse tree.\\\\
    \textit{Answer}:\\
    G =
    \begin{itemize}
      \item $S \rightarrow ABC$
      \item $A \rightarrow Bd\; |\; h$
      \item $B \rightarrow ef$
      \item $C \rightarrow gCg\; |\; Ad$
    \end{itemize}

  \end{enumerate}
  
  \hfill\newline
  \item Given the grammar
  \begin{itemize}
    \item $S \rightarrow aS\; |\; SB\; |\; d$
    \item $B \rightarrow Bb\; |\; c$
  \end{itemize}
  Show that this grammar is ambiguous by showing two parse trees for the string aadccb.\\\\
  \textit{Answer}:\\
  One:
  \begin{parsetree}
    (.S.
      `a'
      (.S.
        `a'
        (.S.
          (.S.
            (.S.
              `d'
            )
            (.B.
              `c'
            )
          )
          (.B.
            (.B.
              `c'
            )
            `b'
          )
        )
      )
    )
  \end{parsetree}
  Another one:
  \begin{parsetree}
    (.S.
      (.S.
        (.S.
          `a'
          (.S.
            `a'
            (.s.
              `d'
            )
          )
        )
        (.B.
          `c'
        )
      )
      (.B.
        (.B.
          `c'
        )
        `b'
      )
    )
  \end{parsetree}\\

  \newpage
  \item Our usual expression grammar (Old Faithful)
  \begin{itemize}
    \item $E \rightarrow E + T \; | \; E - T \; | \; T$
    \item $T \rightarrow T * F \; | \; T / F \; | \; F$
    \item $F \rightarrow (E) \; | \; i$
  \end{itemize}
  can't be used as is in a predictive parser because it has left recursion. Someone had the idea that he could remove left recursion from this grammar by changing it to
  \begin{itemize}
    \item $E \rightarrow T + E \; | \; T - E \; | \; T$
    \item $T \rightarrow F * T \; | \; F / T \; | \; F$
    \item $F \rightarrow (E) \; | \; i$
  \end{itemize}
  Show that this isn’t such a good idea by:

  \begin{enumerate}
    \item  Drawing parse trees for $a - b - c$ using both the original grammar and the revised grammar.\\\\
    \textit{Answer}:\\
    Original:
    \begin{parsetree}
      (.E.
        (.E.
          (.E.
          	(.T.
              (.F.
                `$i \rightarrow a$'
              )
            )
          )
          `-'
          (.T.
            (.F.
              `$i \rightarrow b$'
            )
          )
        )
        `-'
        (.T.
          (.F.
            `$i \rightarrow c$'
          )
        )
      )
    \end{parsetree}
    Modified:
    \begin{parsetree}
      (.E.
        (.T.
          (.F.
            `$i \rightarrow a$'
          )
        )
        `-'
        (.E.
          (.T.
            (.F.
              `$i \rightarrow b$'
            )
          )
          `-'
          (.E.
            (.T.
              (.F.
                `$i \rightarrow c$'
              )
            )
          )
        )
      )
    \end{parsetree}\\

    \item  Using the trees as a guide to determine the results of the corresponding computations if $a = 3$, $b = 5$, $c = 8$.\\\\
    \textit{Answer}:\\\\
    Original: $a - b - c \rightarrow (a - b) - c = (3 - 5) - 8 = -10$\\
    Modified: $a - b - c \rightarrow a - (b - c) = 3 - (5 - 8) = 6$
  \end{enumerate}

  \textit{Comment}: Apparently the change to the grammar was not a good idea because it changed the grammar to be a consequentially different one than the original.\\

  \newpage
  \item Find the \textit{FIRST} and \textit{FOLLOW} sets for the grammar
  \begin{itemize}
    \item $S \rightarrow ABC$
    \item $A \rightarrow a \; | \; Cb \; | \; \epsilon$
    \item $B \rightarrow c \; | \; dA \; | \; \epsilon$
    \item $C \rightarrow e \; | \; f$
  \end{itemize}

  \textit{Answer}:\\\\
  \textit{First Sets}:\\
  Recall the rules for constructing the first set:
  \begin{itemize}
    \item If $X$ is a terminal, then $First(X) = \{x\}$
    \item If a production $X \rightarrow \epsilon$ exists, then $\epsilon \in First(X)$
    \item For non-terminal $X$ and symbols (terminal and non-terminal) $Y_{1}...Y_{k}$, if a production of  $X \rightarrow Y_{1}...Y_{k}$ exists, then $First(Y_{1}...Y_{k}) \in First(X)$
    \item $First(Y_{1}...Y_{k})$ is equal to $First(Y_{1}) \cup ... \cup First(Y_{i}) - \{\epsilon\}$, where $i$ indicates the first $Y$ s.t. $\epsilon \not\in First(Y_{i})$. If there is no such $i$, then $First(Y_{1}...Y_{k}) = First(Y_{1}) \cup ... \cup First(Y_{k}) \cup \{\epsilon\}$
  \end{itemize}
  \begin{align*}
    First(S \rightarrow ABC) =& \{\} \cup First(ABC)\\
                             =& First(A) \cup First(B) \cup First(C) \; - \; \{\epsilon\}\\
                             =& \{\} \cup \{\} \cup \{\} - \{\epsilon\}\\
                             =& \{a, c, d, e, f\}
  \end{align*}
  \begin{align*}
    First(S) =& \{a, c, d, e, f\}
  \end{align*}

  \begin{align*}
    First(A \rightarrow a) =& First(a)\\
                           =& \{a\}
  \end{align*}
  \begin{align*}
    First(A \rightarrow Cb) =& First(C)\\
                            =& \{e, f\}
  \end{align*}
  \begin{align*}
    First(A \rightarrow \epsilon) =& \{\epsilon\}
  \end{align*}
  \begin{align*}
    First(A) =& \{a\} \cup \{e, f\} \cup \{\epsilon\}\\
             =& \{a, e, f, \epsilon\}
  \end{align*}

  \begin{align*}
    First(B \rightarrow c) =& First(c)\\
                           =& \{c\}
  \end{align*}
  \begin{align*}
    First(B \rightarrow dA) =& First(d)\\
                            =& \{d\}
  \end{align*}
  \begin{align*}
    First(B \rightarrow \epsilon) =& \{\epsilon\}
  \end{align*}
  \begin{align*}
    First(B) =& First(c) \cup First(dA) \cup \{\epsilon\}\\
    		 =& \{c\} \cup First(d) \cup \{\epsilon\}\\
    		 =& \{c\} \cup \{d\} \cup \{\epsilon\}\\
             =& \{c, d, \epsilon\}
  \end{align*}

  \begin{align*}
    First(C \rightarrow e) =& First(e)\\
                           =& \{e\}
  \end{align*}
  \begin{align*}
    First(C \rightarrow f) =& First(f)\\
                           =& \{f\}
  \end{align*}
  \begin{align*}
    First(C) =& First(e) \cup First(f)\\
    		 =& \{e\} \cup \{f\}\\
             =& \{e, f\}
  \end{align*}

  \begin{align*}
    First(a) =& \{a\}\\
    First(b) =& \{b\}\\
    First(c) =& \{c\}\\
    First(d) =& \{d\}\\
    First(e) =& \{e\}\\
    First(f) =& \{f\}\\
  \end{align*}

  \textit{Follow Sets}:\\
  Recall the rules for constructing the follow set:
  \begin{itemize}
    \item The $Follow Set$ for terminals is undefined.
    \item $\$ \in Follow(S)$, where $\$$ is the input termination symbol and $S$ is the start symbol of the grammar
    \item If $X$ appears in the RHS of a production of the form $Y \rightarrow aXb$, where $a$ is an arbitrary string (including $\epsilon$) and $b$ is an arbitrary symbol (terminal or not), then $First(b) - \{\epsilon\} \in Follow(X)$
    \item If $X$ appears in the RHS of a production of the form $Y \rightarrow aXb$ and $\epsilon \in First(b)$, then $Follow(Y) \in Follow(X)$
    \item If $X$ appears in the RHS of a production of the form $Y \rightarrow aX$, then $Follow(Y) \in Follow(X)$
  \end{itemize}
  \begin{align*}
    Follow(S: S \; is \; start) =& \{\$\}
  \end{align*}
  \begin{align*}
    Follow(S) =& \{\$\}
  \end{align*}

  \begin{align*}
    Follow(A: S \rightarrow ABC) =& First(BC) - \{\epsilon\}\\
                                 =& First(B) \cup First(C) - \{\epsilon\}\\
                                 =& \{c, d, \epsilon\} \cup \{e, f\} - \{\epsilon\}\\
                                 =& \{c, d, e, f\}
  \end{align*}
  \begin{align*}
    Follow(A: B \rightarrow dA) =& Follow(B)\\
                                =& \{e, f\}
  \end{align*}
  \begin{align*}
    Follow(A) =& \{c, d, e, f\} \cup \{e, f\}\\
              =& \{c, d, e, f\}
  \end{align*}

  \begin{align*}
    Follow(B: S \rightarrow ABC) =& First(C) - \{\epsilon\}\\
                                 =& \{e, f\} - \{\epsilon\}\\
                                 =& \{e, f\}
  \end{align*}
  \begin{align*}
    Follow(B) =& \{e, f\}
  \end{align*}

  \begin{align*}
    Follow(C: S \rightarrow ABC) =& Follow(S)\\
                                 =& \{\$\}
  \end{align*}
  \begin{align*}
    Follow(C: A \rightarrow Cb) =& First(b)\\
                                =& \{b\}
  \end{align*}
  \begin{align*}
    Follow(C) =& \{\$\} \cup \{b\}\\
              =& \{\$, b\}
  \end{align*}
\end{enumerate}

\end{document}
