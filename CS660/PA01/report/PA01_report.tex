\documentclass{article}
\usepackage[utf8]{inputenc}
\usepackage{graphicx}
	\DeclareGraphicsExtensions{.png, .jpeg}
\usepackage[top=1in, bottom=1in, left=1in, right=1in]{geometry}

\title{Compiler Construction \\ PA01: Simple C Programs}
\author{Terence Henriod}
\date{\today}

\begin{document}

\clearpage            % All of
\maketitle            % this,
\thispagestyle{empty} % removes the page number from the title page

\begin{abstract}
\noindent This is a demonstration of some simple C programs that are ideal candidates for
testing a basic C compiler.
\end{abstract}

\newpage
\section{The Code}
\subsection{Hello World!}
\begin{verbatim}
#include <stdio.h>

int
main(int argc, char** argv) {
  printf("Hello World!\n");
  return 0;
}
\end{verbatim}

\subsection{Bubble Sort}
\begin{verbatim}
const int n_items = 5;

void
bubble_sort(int* items, int num_items);

int
main(int argc, char** argv) {
  int* items = (int*) malloc(n_items * sizeof(int));
  int i;
  for (i = 0; i < n_items; i++) {
    items[i] = n_items - i;
  }

  bubble_sort(items, n_items);

  free(items);
  items = 0;

  return 0;
}


void
bubble_sort(int* items, int num_items) {
  // This is the 'textbook' implementation and not the awesome optimized one
  int i = 0;
  int j = 0;
  int temp;

  for (i = 0; i < n_items; i++) {
    for (j = i + 1; j < n_items; j++) {
      if (items[i] > items[j]) {
        temp = items[i];
        items[i] = items[j];
        items[j] = temp;
      }
    }
  }
}
\end{verbatim}

\newpage
\subsection{Iterative Factorial}
\begin{verbatim}
int
main(int argc, char** argv) {
  int x = 5;

  int result = iterative_factorial(x);

  return 0;
}

int
iterative_factorial(int x) {
  int result = 1;

  if (x < 0) {
    result = -1;
  } else {
    while (x > 1) {
      result *= x;
      x--;
    }
  }

  return result;
}
\end{verbatim}

\subsection{Recursive Factorial}
\begin{verbatim}
int
main(int argc, char** argv) {
  int x = 5;

  int result = recursive_factorial(x);

  return 0;
}

int
recursive_factorial(int x) {
  int result;

  if (x < 0) {
    result = -1;
  } else if (x <= 1) {
    result = 1;
  } else {
    result = x * recursive_factorial(x - 1);
  }

  return result;
}
\end{verbatim}

\end{document}

