\documentclass{article}
\usepackage[utf8]{inputenc}
\usepackage[top=1in, bottom=1in, left=1in, right=1in]{geometry}
\usepackage{indentfirst}

\title{CS791x \\ Week02, Day01: Experimental Robotics}
\author{Terence Henriod}
\date{\today}

\begin{document}

\maketitle

% \begin{abstract}
% Experimental Robotics.
% \end{abstract}


\newpage
\section{Robots in Organizations: The Role of Workflow, Social, and Environmental Factors in Human-Robot Interaction}

\subsection{Focus of the Paper}
The study presented in the paper aimed to study the reception of integration of robots in the workplace, in different work settings. The authors claim that their findings led to a set of helpful guidelines for the design of robots for use in organizations.

\subsection{Experiment Design}
The method used in the paper was an ethnographic study. Although three types of data were collected, only the latter two were used in the results of the paper. The data collection methods were:
\begin{itemize}
  \item Participant Observations (not used in results; only used with the robot manufacturer's staff)
  \item Fly-on-the-Wall Observations
  \item Interviews of Staff (interviews of the hospital staff were more structured than those of the robot manufacturer's staff)
\end{itemize}

It should be noted that some of the staff interviews at the hospital were the added due to the presence of surprising results.

Once data was collected, the data was coded to allow for analysis of the data, which might be otherwise considered qualitative.

\subsection{Results}

The observational results of the paper focus primarily on the differences between the attitudes and interactions with the robot for the medical and post-partum units of the hospital. In short, the robot was very much disliked, and even abused, in the medical unit, while the robot was favored in the post-partum unit. Greater acceptance of the robot by staff members led to better utilization of the robot for tasks. 

\subsection{Strengths}

The paper did a good job of introducing the ideas behind the research and including related work in the area of research. A god job was done, pointing out that there may be organizational ``requirements" for a technology (in terms of things like social dynamics) that might not be in the specification for the technology, but that these factors can be important to the adoption of a technology into an organization.

Also, the paper was able to report some important observations and factors of design, even if it was stating the obvious somewhat. It was good to call attention to the fact that emotional tone of staff can affect their acceptance of the robot (perhaps research should be done do explore if all of the staff would be equally accepting of the robots, attitudes aside) and that while some members of an organization might really benefit from a technology, it may increase the burden on other members.

The study of an organization that had experienced long term use (years of continuous use, as opposed to occasional experiences) of the robots was also a strength of this study.

\subsection{Weaknesses}

The first weakness, in my opinion, is that the ``fly on the wall" observations were not very secretive. If I understand the paper correctly, these observations involved a researcher being physically present to view the activities of the organizational staff and the robot, which might have introduced unexpected effects. If one is going to claim to have done ``fly on the wall" observations, it is probably better form to use cameras (that are possibly hidden) to better capture natural interactions between staff and the robot.

It is also my suspicion that since a researcher had to be present, a full sampling of the interactions between staff and the robots was not captured. I suspect that observations from the early hours of the morning, or the complete course of a staff member's shift were fully observed.

A second weakness might be that in the ``Design Implications" section, several recommendations are made, that on their face seem to be good recommendations, but do not seem to reference observations from the study. They seem like passable ideas that were just added to the paper - they don't seem to have arisen from the research, which, if they did not, undermines the call for investigation of the finer workings of an organization made in the ``Related Work" section. It should be clear that recommendations are a product of the research.


\section{An affective guide robot in a shopping mall.}

\subsection{Focus of the Paper}
The field study conducted in the paper aimed to explore the use of a robot for social interaction in a busy place. The authors sought to find out if a robot could provide information in an open, public environment; if a robot could ``influence people's daily activities," and if people would spontaneously and repeatedly interact with a robot.

\subsection{Experiment Design}
A robot, designated \emph{Robovie}, was placed in a mall. The robot was stationary, and used pressure sensitive tiles and RFID tags to detect and recognize nearby people. The robot also used a microphone to capture voice input from mall patrons. A human operator was used to mitigate the technological shortcomings of Robovie (which was legitimate, as the issues solved by a human operator are indeed issues that are difficult to solve in their own right).

In order to interact with Robovie, shoppers (participants) had to register with the robot's keepers and were given an RFID tag (that they were allowed to keep) so that Robovie could identify them.

Robovie used a human operator to perform translation from participant speech/requests to text for Robovie. Robovie functioned both in a utility capacity, providing directions and recommendations to shoppers, and in a social capacity, engaging customers in conversation, ``remembering" details about shoppers, and sharing information about itself.

Data was collected in the form of survey responses and counts of interactions and conversational exchanges.

\subsection{Results}
The survey results of this experiment were slightly positive, with average survey scores ranging from 4.6 to 5.3 on a scale of 1.0 to 7.0 for each of the categories.

In terms of encouraging repeated interaction, I would say that Robovie was not successful, with the mean number of visits per participant being only two. However, since there were some participants who visited many more times than the typical person (I am referring to the 5 to 18 visit category), then this would suggest that the robot might be effective in it's goal for repeated interaction for certain populations, but not the greater population.

One remark on the topic of repeated interaction I have is that some of the survey results indicated that some of the participants found that the robot's directions and help were very effective, (my assertion:) so why would they need repeated interaction with it once they have the information they needed? To me, this speaks to the utility of the robot system, and people using the robot for its utility may not be interested in its social applications. I would argue that social applications and utility might be at odds in many scenarios.

\subsection{Strengths}
I would judge their system design and use of WOZing to be good. The robot system was designed to operate autonomously, but use a human operator only for situations where the robot's technology was insufficient. This was a good design when you consider the fact that some of the more recent technologies could actually replace the human operator, leaving the system still functional (think about the background noise filtering microphones in cell phones, or the ``personal assistant" technologies that can perform speech recognition and NLP functions like Siri or OK-Google).

\subsection{Weaknesses}
The paper asserts that mall-guide robots could be used as an avenue for advertising for shops within the mall. However, this seems inconsistent, since the paper says that ``Since a robot's presence is novel, it can attract people's attention". If a robot is to become commonplace in malls, it will no longer be novel, and therefore likely that it's advertising potential will be lost. Perhaps the a robot's advertising potential could be recovered by providing a more personalized interaction, such as pointing to shops that might satisfy or be related to a customer's request if they solicit the robot's help.

Much of the robot's intended appeal was on novelty, with many conversation pieces needing to be invented and switched out regularly. This seems to indicate that might not be successful in the long term.

Another weakness that could be mentioned was the timing of when the robot was used, only on weekdays from 1:00 to 5:00. Understandably, this was due to an agreement with the mall owners, but it might prevent scenarios or observations that would make the results more complete. Notably, the busy week and weekends were excluded, which might prove to have very different shoppers or needs. I don't konw if Japan is different than the US, but I would think that shoppers at a mall generally are more relaxed and have more free time to explore the mall.

\section{Using Proprioceptive Sensors for Categorizing Human-Robot Interactions}

\subsection{Focus of the Paper}
This paper focuses on the use of alternative sensors to characterize human interaction with a robot. The papare further asserts that by using these sensors to characterize the different states that the robot may be in, the robot can be augmented to be more effective in its design, in this case, the social engagement of children.

\subsection{Experiment Design}
The ``Roboall" prototype robot as constructed in order to execute the experiment. Roball utilized a 3-axis accelerometer and tilt sensors in order to characterize how it was being handled. The experimenters first characterized the readings as the robot was used in 7 different scenarios. A barrier was erected to enclose a small area to help control the experiment and limit the scope of the environment that the robot would have to experience.
Once the sensor reading classes were characterized, then the Roball was tested in short play sessions with children to verify the classification of the classes of handling that the robot was recieving. Data was recorded by the robot itself, and the sessions were video recorded so that the researchers could validate the states observed in the video with those recorded by the robot.
Finally, Roball was augmented to make different sounds/play different speech samples based on its perceived state.

\subsection{Results}
The Roball was able to indicate the different states that it could be in with varying success. Further, using the ability to identify these states can be used to augment a robot's behavior. The paper remarks that the augmented behavior for the robot was more enjoyable and engaging for the children.

\subsection{Strengths}
The researches attempted to use alternative, cheap, (appear to be) off-the-shelf sensors to help the robot characterize its states. Perhaps this was even a novel attempt at detecting robot state, since most often we are concerned with localization, rather than context.

\subsection{Weaknesses}
This paper has some very serious weaknesses in terms of descriptions of results and methods was lacking. It was not clear how the sensor readings were clustered into the various categories. Further, better methods could have been used to determine what state the Roball was in (perhaps that would be a task for a re-implementer of today).

It might be considered a weakness of the paper that the social implications of the robot were not given full treatment. The paper definitely had more of a feel that the researchers were just seeing if they could build a system to identify context using alternative sensors. For this purpose, the researchers seemed to have done an OK job (maybe a little less than OK...) at building such a system; but their treatment of the importance of touch and social interaction in their pre-amble really left me feeling that more treatment should have been given to the social implications of a robot that could perceive its context through proprioception.

\end{document}